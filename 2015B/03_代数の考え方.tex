\documentclass{jsarticle}
%\usepackage{textcomp}
\title{代数の考え方(10)}
%\author{141-821706-0 松山 和弘}
\author{142-004814-1 松山 和弘}
\date{\today}
\begin{document}
\maketitle

\section{問1}

$a^3+b^3+c^3-3abc$の3根は、

\[ a^3+b^3+c^3-3abc =
    (a+b+c)(a^2+b^2+c^2-ab-bc-ca) =
    (a+b+c)(a+b \omega +c \omega^2)(a+b\omega^2 + c\omega )
\]

\[ 
    \omega = \frac{-1 + \sqrt{3}i}{2}
\]


\[ 
    x=-b-c
\]

\[ 
    x=-b\omega-c\omega^2
\]

\[
    x=-b\omega^2-c\omega
\]

となる。よって、
$a^3-3abc+b^3+c^3$を、$x^3+px+q$と比べて
\[
    p=-3bc
\]

\[
    q=b^3+c^3
\]

となる$b,c$が決まれば3根がわかる。

\[
    -3bc=p より、
\]

\[
    b^3c^3=-\frac{p^3}{27}
\]

となる。これは、2次方程式の式と根の関係より、

\[
    (t-b^3)(t-c^3)=t^2-qt- \frac{p^3}{27}=0
\]
となる。これより、$b^3,c^3$についての判別式は、

\[
  D_{b^3c^3}=27q^2+4p^3
\]

となる。解は以下となり、平方根中に判別式  $D_{b^3c^3}$ が含まれていることを確認できる。

\[ x=\left(
 \frac{\sqrt{27\,q^2+4\,p^3}}{2\,3^{\frac{3}{2}}}-\frac{q}{2}\right)
 ^{\frac{1}{3}}-\frac{p}{3\,\left(\frac{\sqrt{27\,q^2+4\,p^3}}{2\,3^{
 \frac{3}{2}}}-\frac{q}{2}\right)^{\frac{1}{3}}} 
\]


\[ 
 x=\left(\frac{\sqrt{3}\,i}{2}- \frac{1}{2}\right)\,
 \left(\frac{\sqrt{27\,q^2+4\,p^3}}{2\,3^{\frac{3
 }{2}}}-\frac{q}{2}\right)^{\frac{1}{3}}-\frac{\left(-\frac{\sqrt{3}
 \,i}{2}-\frac{1}{2}\right)\,p}{3\,\left(\frac{\sqrt{27\,q^2+4\,p^3}
 }{2\,3^{\frac{3}{2}}}-\frac{q}{2}\right)^{\frac{1}{3}}}
\]


\[
 x=\left(-\frac{\sqrt{3}\,i}{2}-\frac{1}{2}\right)\,
 \left(
 \frac{\sqrt{27\,q^2+4\,p^3}}{2\,3^{\frac{3}{2}}}-\frac{q}{2}\right)
 ^{\frac{1}{3}}-\frac{\left(\frac{\sqrt{3}\,i}{2}-\frac{1}{2}\right)
 \,p}{3\,\left(\frac{\sqrt{27\,q^2+4\,p^3}}{2\,3^{\frac{3}{2}}}-
 \frac{q}{2}\right)^{\frac{1}{3}}}
\]

判別式$D$は以下となる。

\[
  D=(-1)^{3(3-1/2)}D_{b^3c^3}=-27q^2-4p^3
\]



\section{問2}
$x^5=1$は、「円周の5等分」問題である。
オイラーの公式により5箇の根がある。

\[x = \cos 2 \pi \frac{0}{5}+i \sin 2 \pi \frac{0}{5} = 1\]

\[x = \cos 2 \pi \frac{1}{5}+i \sin 2 \pi \frac{1}{5} \]

\[x = \cos 2 \pi \frac{2}{5}+i \sin 2 \pi \frac{2}{5} \]

\[x = \cos 2 \pi \frac{3}{5}+i \sin 2 \pi \frac{3}{5} \]

\[x = \cos 2 \pi \frac{4}{5}+i \sin 2 \pi \frac{4}{5} \]

$ 72^\circ $は、円の$\frac{1}{5}$であるため、 $ \cos72^\circ, \sin72^\circ $ は以下となる。

\[ \cos72^\circ =  \cos 2 \pi \frac{1}{5} \]
\[ \sin72^\circ =   \sin 2 \pi \frac{1}{5} \]

\section{問3}

\[z = x + iy = (a + bi)^2 = a^2 + 2abi - b^2\]

としたとき、

\[x=a^2-b^2\]
\[y=2ab\]

となる。

\[x = a^2-b^2 = a^2 - \frac{y^2}{4a^2}  \quad より \quad 4a^4-4xa^2-y^2=0  \]

a,bは、

\[ a^2 = \frac{x\pm \sqrt{x^2+y^2}}{2}\]

\[ a = \pm \frac{\sqrt{x+ \sqrt{x^2+y^2}}}{\sqrt{2}}\]

\[ b = \frac{y}{2a} =
   \pm \frac{y}{2} \frac{\sqrt{2}}{\sqrt{x+\sqrt{x^2+y^2}}} =
   \pm \frac{y}{\sqrt{y^2}} \frac{\sqrt{-x+\sqrt{x^2+y^2}}}{\sqrt{2}}
  \]

となる。
よって、

\[ \sqrt{z} = \sqrt{x+iy} = a+bi =\pm \frac{\sqrt{x+ \sqrt{x^2+y^2}} + \sqrt{-x+ \sqrt{x^2+y^2}}i}
  {\sqrt{2}}  \qquad (y>0)\]


\[ \sqrt{z} = \sqrt{x+iy} = a+bi= \pm \frac{\sqrt{x+ \sqrt{x^2+y^2}} - \sqrt{-x+ \sqrt{x^2+y^2}}i}
  {\sqrt{2}}  \qquad (y < 0)\]

となる。

\end{document}

