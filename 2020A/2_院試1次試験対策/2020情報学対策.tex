\documentclass[12pt]{jsarticle}
\title{
情報学
論述問題対策
}
\author{松山 和弘}
\date{\today}
\begin{document}
\maketitle

\section{現代社会が抱える課題と情報通信技術(ICT)の活用}

\subsection{
統計や機械学習を社会実装する場合のバイアス問題
}

\begin{quotation}
現代社会が抱える課題を1つ取り上げ、
その課題を解決するために、
どのように情報通信技術(ICT)が活用できるかを論じなさい。
\end{quotation}

ビジネスの現場で、
統計や機械学習を社会実装する場合、
結果の厳密さよりも
成果を出すことが重要と言われることがある。

ビジネス上求められる結果が出るように、
統計データの採取方法をねじ曲げたり、
機械学習のための学習データが偏よってると疑われる
ことが行われている。

例えば、
採用に関するAI、
検索エンジンや画像分類の
事前学習に人種差別が含まれている疑いがあることがある。
事前学習の内容はブラックボックスとなっており、
内容を確認することができなくなっている。

このようなバイアスは、
人間でも発生する(社会心理学での認知のバイアス)。

統計や機械学習を行うのは人間であり、
人間のバイアスがAIで処理されてしまう問題がある。

このような問題に陥らないために、
統計学、データサイエンスの視点を持ち、
データの適切さに問題意識を持つことは重要と考える。

\subsection{
統計や機械学習を社会実装する場合の過学習問題
}

\begin{quotation}
現代社会が抱える課題を1つ取り上げ、
その課題を解決するために、
どのように情報通信技術(ICT)が活用できるかを論じなさい。
\end{quotation}

ビジネスの現場で、
統計や機械学習を社会実装する場合、
結果の厳密さよりも
成果を出すことが重要と言われることがある。

ビジネス上求められる結果が出るように、
統計データの採取方法をねじ曲げたり、
機械学習のための学習データが偏よってると疑われる
ことが行われている。

機械学習では、
偏りのあるデータを強調して学習してしまう、
過学習が発生しがちであり、
これを防ぐために、
学習内容が偏らないようにして、
汎化能力を持たせる必要がある。

学習内容が偏らないようにして、
過学習、汎化能力を持たせる必要があるのは、
AIだけではなく、人間の学習についても同様と考えられる。

例えば、ビジネスの現場では、
統計データを採取しながら、
PDCAサイクルを回して、
ビジネス上の成果を上げようとするが、
結果に結びつかないケースがあるようだ。
これは、 機械学習での過学習と同様なことが、
人間が行うビジネス上のプロセスでも発生していると考えるのが
適切かもしれない。

ビジネスの現場では、人間や機械学習に関する統計処理で、
過学習や汎化能力の劣化が起きたとしても、
認知されないことがあると考えられる。
このような問題に陥らないために、
統計学、データサイエンスの視点を持つことは重要と考える。


\subsection{デジタル庁創設について}

\begin{quotation}
現代社会が抱える課題を1つ取り上げ、
その課題を解決するために、
どのように情報通信技術(ICT)が活用できるかを論じなさい。
\end{quotation}

COVID-19感染拡大により、社会が変容する中、
以下の課題が浮き彫りとなった。

\begin{enumerate}
\item 行政手続きの遅さや連携不足により、現金給付の遅延が発生した。
\item 関係機関間での情報のやりとりにFAXが使われており、
    情報伝達の相互運用性(interoperability)に問題があった。
\item 押印手続等のでテレワーク阻害要因がある。
\item 教育関連では、オンライン教育に必要な基盤やノウハウ不足していた。
\end{enumerate}


菅政権の方針として、
デジタル化推進のための、
「デジタル庁」を新設を目指している。
各省庁ごとに分散しているデジタル化の一元化を行う。

現在、
河野太郎行政改革相主導による、
行政手続きでの捺印廃止が進められている。
これには、
COVID-19のためのテレワークであるのに、
捺印のために出社が必要となることが多く発生したため、
問題視されていたことによる。

さらに、書面、FAXも廃止が検討される見込みである。

情報通信技術(ICT)としては、
電子署名、ブロックチェーン
が捺印のかわりや書面の出処の確認に使用することができるが、
これまで国内での適用は限定的であった。
また、クラウド化により、クラウド上のタスクの連携、
クラウドのファイルシステムの使用により、
相互運用性向上を見込むことができる。

教育関連では、
適切なシステムを導入し、
オンライン化のノウハウを蓄積していく必要がある。
放送大学は、リファレンスにすべきと考える。


\subsection{DXと2025年の崖}

\begin{quotation}
現代社会が抱える課題を1つ取り上げ、
その課題を解決するために、
どのように情報通信技術(ICT)が活用できるかを論じなさい。
\end{quotation}

DX(デジタルトランスフォーメーション)とは、
ビジネスモデル、
文化、
組織、
制度
といった企業を変革していく一連の取り組みである。

経済産業省が2019年9月に公表したDXレポート(2025年の崖)では、
DX(デジタル化の推進)より以下のレガシーシステムやIT人材不足を問題としている。

\begin{enumerate}
\item 21年以上稼働しているレガシーシステムが全体の6割となり、
      IT予算の9割が保守に費やされる。
\item レガシーシステムの老朽化によりサポート不可能となる。
\item レガシーシステムがブラックボックス化しコントロール不能となる。
\item IT人材が退職等により不足する。
\end{enumerate}

ブラックボックス化したレガシーシステムの移行には困難が伴い、
泥沼化することがある(京都市の事例など)。

IT人材供給のピークは、2019年と言われており、
既に人材不足が始まっている。
今後のソフトウェア開発では、
これまで確保できていた人員を確保できなくなってきている。

これまでは、カタログスペックを良く見せるための
使われない機能のための開発や、
実際には参照されることのない文書作成や、
一時的なプレゼンテーションにしか使われない文書作成に、
多くのリソースが注ぎ込まれてきた。
これは、IT人材不足下では見直すべきであろう。

最近の情報通信技術(ICT)である
AIやデータマイニングの活用が期待されているが、
効果や実現性に疑問があるものがある。
開発リソースの投入は、慎重であるべきだろう。

DXで要求されるのは、
業務プロセスを見直し、仕様を明確にした上で、
限られた開発要員で無理のないように、
システムを構築することである。
そして、将来の保守性を考慮し、特殊な仕様は避けるべきであろう。

\begin{quotation}
京都市の30年可動するメインフレームで動作する
基幹システムのバッチ処理をオープンシステムに移行する開発を、
2014年に始めたが、2017年に稼動ずことができず、
開発会社を変更して再開発を行なったが、
2020年1月に稼働することができず、2度の失敗となっている。

失敗の原因は、レガシーシステムのブラックボックス化と言われており、
旧バッチ処理システムの結果を新オープンシステムの結果と一致させるために、
テスト数が膨大となっていることが関係していると言われている。

ブラックボックス化したレガシーシステムを
新オープンシステムの結果一致を要件としているのは、
業務プロセスの見直しが行われなかったことを意味している。
本来は、業務プロセスを見直し、仕様を明確にした上で、
旧システムと結果比較しなくても良いように開発を行うべきであったと考えられる。
\end{quotation}


\subsection{COVID-19と情報}

\begin{quotation}
現代社会が抱える課題を1つ取り上げ、
その課題を解決するために、
どのように情報通信技術(ICT)が活用できるかを論じなさい。
ただし、
下記のキーワードを2つ以上用いて800字以内で記述すること。

キーワード:
アルゴリズム,
顔認証,
計算量,
コンピュータビジョン,
自然言語処理,
情報操作,
情報デザイン,
人口知能(AI),
制御,
相互運用性,
データサイエンス,
データベース,
総合認証(SSD),
フィルターバブル,
モノのインターネット(IoT)
\end{quotation}

本来であれば、
感染状況について、統計的に適切に情報が扱われ、
正確な情報を皆が把握して、合理的な政治判断や、
リソースの適切な分配や利用が求められる。
そして、ICTはこれを支援する仕組みを提供することが求められる。

感染状況を把握するためには、
PCR検査対象のサンプリングや検査結果の精度について、
適切な統計手法で扱う必要があり、
時系列的に適切にデータを扱わなくてはならず、
世界的な統計データと比較検討できることが望まれる。

これらのデータを遅延なく収集と集計できるように支援できる物でなくてはならない。
適切なデータベース化等が必要だろう。

実際は、
自治体ごとに、検査件数、感染者数、死亡者数の公表形式が統一されておらず、
検査件数が公表されなかったり、時系列的にズレていたりして、
統計的に不適切にデータが扱われていた。
保健所から自治体への報告にはFAXが使われており、
効率的なデータの伝達は行われていたとは言い難い。

統計に関する問題以外では、
正確でなはい不正確な情報や差別等を含む不適切な情報がマスコミやSNS等で流され、
情報操作や、
不適切な政治判断、
医療関連リソースの不適切な使用が行われていると思われる。

COVID-19関連で、
情報通信技術(ICT)が適切に使われたかどうかは、
今後の検証を要すると思われる。


\subsection{
少子高齢化による人口減少とソフトウェア開発
}

\begin{quotation}
現代社会が抱える課題を1つ取り上げ、
その課題を解決するために、
どのように情報通信技術(ICT)が活用できるかを論じなさい。
ただし、
下記のキーワードを2つ以上用いて800字以内で記述すること。

キーワード:
オープンデーター,
音声認識技術,
学習,
拡張現実(VR),
仮想化,
学校経営,
国際化,
人口,
セキュリティ,
ソフトウェア開発,
地域振興,
データマイニング,
\end{quotation}

経済産業省が2019年9月に公表したDXレポート(2025年の崖)では、
デジタル化の推進より以下のレガシーシステムやIT人材不足を問題としている。

\begin{enumerate}
\item 21年以上稼働しているレガシーシステムが全体の6割となり、
      IT予算の9割が保守に費やされてしまう。
\item レガシーシステムが老朽化によりサポート不可能となる。
\item レガシーシステムのブラックボックス化しコントロール不能となる。
\item IT人材が退職等により不足する。
\end{enumerate}

一時期、AIの発展により失う職業の候補に、
ソフトウェア開発があり、
ソフトウェア開発要員の減少の影響は無いかのように喧伝
されたことがあり、
これがミスリードとなっていたと考えられる。

IT人材供給のピークは、2019年と言われており、
既に、ベテランのTI人材の退職と人材の供給不足により、
人材不足が始まっている。

さらに、
介護離職の増加、
レガシーシステム保守の増大、
AI等の新技術習得の困難さ
が加わり、
2025年以降、IT人材不足はさらに悪化すると見られている。

2025年の崖として
この問題が認識されるのは、
つい最近であり、
状況が悪化し始めてからの対策となっている。

2025年の崖の対策として求められるのは、
業務プロセスの変革を進めながら、
限られた開発要員で無理のないように、
速やかにデジタル化を進めることである。

これは、
既存のレガシーシステムをそのまま移行することではないこと
がポイントとなる。

\subsection{
クラウドコンピューティングと個人情報保護
}

\begin{quotation}
現代社会が抱える課題を1つ取り上げ、
その課題を解決するために、
どのように情報通信技術(ICT)が活用できるかを論じなさい。
ただし、
下記のキーワードを2つ以上用いて800字以内で記述すること。

キーワード:
個人情報保護,
動画配信,
ドローン,
クラウドファンディング,
人口知能,
ビッグデータ,
センサネットワーク,
セキュリティ,
データサイエンス,
拡張現実,
労働,
ユニバーサルデザイン,
プログラミング教育,
ソーシャルネットワーキングサービス(SNS),
観光
\end{quotation}

2013年6月、
エドワード・スノーデンにより、
国際的監視網(PRISM)が実在することについての、
内部告発が行われた。

2020年9月2日、
サンフランシスコ連邦高等裁判所は、
エドワード・スノーデンが暴露した、
米国家安全保障局(NSA)による大量監視を違法とする判決を下した。


現在、
クラウドコンピューティングの利用が一般化しており、
企業では、文書等をクラウド上に置いて共有することが可能となり、
個人でも、スマートフォン等のデータをクラウドに保存することは、
日常的に行われている。
これにより、
クライアント端末のシンクライアント化が可能となり、
PC運用の手間は省けるようになり、データの共有も容易となった。

クラウド上のデータは、クラウド側で、
更新や参照者の履歴を保存し監視することができる。
また、検索の容易性確保のため、
データの内容を解析してラベルをつけることができる。
クラウド上の画像データが自動で分類される仕組はこれである。

検索の容易性確保のためのラベル付けのみであれば、
合法的であり問題ないが、
用途を誤れば、検閲や個人情報の盗用となり問題である。

クラウド上のSNS等の情報は、個人情報保護され、
ていなければいけないはずである。
クラウドサービスを提供する側は、
個人情報を不正利用することは技術的に可能であるが、
倫理的に問題である。

今後もセキュリティ確保のためには、
エドワード・スノーデンの言うことに、
耳を傾けるべきだろう。


\section{
情報基盤技術の役割
}

\subsection{
クラウドコンピューティング
}

\begin{quotation}
情報基盤技術は現代社会において大きな役割を果たしている。
あなたが注目する情報基盤技術の具体例を一つあげ、
それが現代社会においてどのような役割を果たしているか、
800字以内で論じなさい。
\end{quotation}

現在、
クラウドの利用が一般化しており、
企業では、文書等をクラウド上に置いて共有することが可能となり、
個人でも、スマートフォン等のデータをクラウドに保存することは、
日常的に行われている。
これにより、
クライアント端末のシンクライアント化が可能となり、
PC運用の手間は省けるようになり、データの共有も容易となった。

企業のタスクをクラウドサーバ上で行うことにより、
自社にサーバを設置する必要がなくなり、
サーバ管理をアウトソーシングすることが可能となった。
また、タスクの増減に応じて計算リソースの動的割り当てが可能となり、
リソースの使用量のみ課金という形態が可能となった。

クラウド化により、
システム間の相互運用性(interoperability)を向上することができる。
これは、
シングルサインオン(SSO)ユーザ認証、
クラウドのファイルシステムやDBの使用により、
クラウド上のタスク間のデータ受け渡し機能等による。

また、
クラウドベンダが提供する機械学習用や、
大規模数値計算向け計算リソースへのアクセスが容易となっている。

以上により、
クラウドコンピューティングにより、
コンピュータのリソースの使い方に柔軟性を与え、
コンピュータを使用する上での利便性は向上した。

クラウド上のデータは、クラウド側で、
更新や参照者の履歴を保存し監視することができる。
また、検索の容易性確保のため、
データの内容を解析してラベルをつけることができる。
クラウドサーバ上の画像データが自動で分類される仕組はこれである。

検索の容易性確保のためのラベル付けのみであれば、
合法的であり問題ないが、
用途を誤れば、検閲や個人情報の盗用となり問題である。

クラウド上のSNS等の情報は、個人情報保護され、
ていなければいけないはずである。
クラウドサービスを提供する側は、
個人情報を不正利用することは技術的に可能であるが、
倫理的に問題である。

今後もセキュリティ確保のためには、
エドワード・スノーデンの言うことに、
耳を傾けるべきだろう。
\section{ヒューマン分野}

\subsection{働き方改革}

\begin{quotation}
現在、 
情報通信技術(ICT)
を活用した働き方改革が大きな話題を集めている。
働き方改革が求められている理由を説明した上で、
ICTの利活用によって、
どのような働き方が実現可能なのか、
具体的な例をあげながら800字以内で論じなさい。
\end{quotation}

2015年発表の国勢調査結果で、
日本の人口減少が確認された。
今後、
少子高齢化に伴い生産年齢人口の減少する。
また、
高齢者の高齢化が進み、
介護離職が増加して労働人口が減少していく。
少子化問題は、
1990年頃から認識されていたが、
これまで適切な対策が行われず、
事態は放置されてきた。
出生率を上げるのは、
既に困難な状況となっている。

状況が悪化した状態ではできることは限られるが、
現在進められている働き方改革の方針は、
以下となっている。

\begin{enumerate}
\item 一億総活躍社会
\item 多様な働き方
\item 格差の固定化を回避
\item 成長と分配の好循環を実現
\end{enumerate}

ここでは、
「一億総活躍社会」、
「多様な働き方」
についてのICTの利活用に触れる。

「一億総活躍社会」では、
働ける人は働けるように支援する仕組みが必要となる。
ニートなどの失業者や高齢者に対する職業教育と就職支援。
非正規労働者のキャリアアップ支援。
就労者に対しては、リカレント教育を進めていく必要がある。
リカレント教育の例としては、企業向けデータサイエンス教育
が最近注目されている。

ICTは、
これらの教育や就職を支援するために利用されなくてはならない。
放送大学の「データサイエンスプラン」は、
リカレント教育としてのICTの利活用の参考となる。

「多様な働き方」で必要なのは、
まず業務プロセスの見直しである。
例えば、リモートワークが可能となるように、
捺印のために出社しなくても業務が完結できる仕組みにする必要がある。
会議の頻度とか見直す必要があるだろう。
ICTの利活用として、
リモート会議にzoom等が使われているが、
これだけでは不十分である。
業務プロセスの見直しを行い、ICTを導入すべきである。

労働人口減少対応のために、
AIやロボット導入とか言われることがあるが、
実現性が疑わしいものがある。
導入には慎重であるべきだろう。


\subsection{電子決済のトラブル}

\begin{quotation}
今日、
インターネットに関わる様々なトラブルが社会問題となっています。
トラブルの具体例を2つあげ、
そうしたトラブルを防ぐために、
個人と社会はどのような取り組みを行なっていくべきか、
1200字以内で論じなさい。
\end{quotation}

銀行口座とクレジットカードの電子決済のトラブルについて論じる。

NTTドコモの電子決済サービス(ドコモ口座)は、
連携する銀行の口座を登録し、
預金からチャージして、支払いできるサービスである。

ところが、銀行口座保有者が知らないうちに、
ドコモ口座の登録ができてしまい、
勝手に預金が引き出される事態が発生してしまった。

ドコモ口座は、本人出なくても、口座番号、暗証番号
が合えば本人でなくても口座登録できてしまう。
これには、暗証番号固定で口座番号を変えながら攻撃する
「リバースブルートフォース攻撃」
が使われたのではないかと言われている。
これは、高度な攻撃手法を使用しなくてもも、
銀行口座の乗っ取りは可能となっている。

ドコモ口座以外Pay系送金でも同様な問題が発生している。
また、ゆうちょ銀行のデビットカード間の送金でも、
同様な問題が発生している。

問題発生後、
NTTドコモやゆうちょ銀行の関係者が記者会見
を行なっているが、
会見内容は要領を得ておらず、
説明責任に問題ある展開となっている。

これまで、クレジットカード番号が盗まれて、
勝手に使用される問題は発生していたが、
注意喚起は行われており周知の問題となっていた。
クレジットカードの不正利用は、不正利用と手数料を
バランスさせることで対応しており、
不正利用自体は発生している。

ドコモ口座の問題は、
銀行口座はクレジットカードのように勝手に使用されることはないだろうと
信用されていたが、そうではないとなったことだ。

また、銀行口座に設定されている引き落とし設定を、
口座保有者が確認できないことは問題であり、
仕組みを見直す必要がある。

電子決済のトラブルの件で、
クレジットカードだけではなく、
銀行口座であっても、
金が消えることがある。
結局、他者にお金を預けることにはリスクがあることを、
認知しておくべきだろう。

本件は、金融システムの信頼性に関わる問題であり、
責任を明確にして、
被害者は全て保証されるべきであると考える。

\subsection{
障がいのある人の学習
}

\begin{quotation}
情報技術の進展が、
障がいのある人の生活や学習スタイルにどのような影響を与えるかについて、
800字以内で述べなさい。
\end{quotation}

視覚障害や聴覚障害があると、
外部からの情報取得が不足し、
生活や学習に支障をきたす。
情報技術の進展により、
この情報不足を補うことが期待される。

最近のAI(機械学習)の進展により、
音声認識、画像認識の性能が向上し、
スマートフォンのような小型の機材で
高度な処理を行うことが可能となった。

これにより、自動字幕のような、
音声のテキスト化や、
自然に近い音声による文章読み上げができるように
なってた。

今後、スマートフォンなどのSoCに搭載される
機械学習用プロセッサーやGPUの性能向上や、
機械学習のための事前学習データの向上により、
さらなる性能向上を期待することができる。

近い将来、
これらの技術がさらに身近なものになると考えられる。

\subsection{
障がいのある人の生活と介護
}

\begin{quotation}
情報技術の進展が、
障がいのある人の生活や学習スタイルにどのような影響を与えるかについて、
800字以内で述べなさい。
\end{quotation}

今後、
高齢化社会が進む見込みとなっており、
介護を受けるケース、介護を行うケースが増えていく見込みである。
また、認知障患者の増加が予想されており、
介護を受ける側に比較して、介護を行う側の人員不足が予想され、
介護の負担は増大する見込みである。

この状況では、
介護を受ける側、介護を行う側双方への支援が必要であり、
情報技術は、これを支援しなくてはならない。

最近は、
身に着けるセンサーが普及してきている。
2020年9月発表のAppla Watchでは、
血中酸素濃度、心電図の計測が可能となっている。
このようなセンサーと通信機能は、
リモート診断やAIを併用した自動診断と合わせることで、
介護を受ける側、介護を行う側双方への支援となるとされている。

介護現場へのロボットの導入が期待されているが、
課題は多く、掃除用等の限定的な範囲でしか実現できてないようだ。
今後の課題である。


\section{マルチメディア分野}

\subsection{Vision API}

\begin{quotation}
日常生活に変化と多様性をもたらしているマルチメディアの事例を取り上げ、
その概要をまとめて記述し、
それらを踏まえた上で、
あなたが考えるマルチメディアの今後の可能性と課題について、
800字以内で論じなさい。
\end{quotation}

\begin{quotation}
マルチメデイア技術は現代社会に劇的な変化をもたらしている。
あなたが注目するマルチメディア技術を1つ挙げ、
それが現代社会に及ぼした
(あるいは及ぼしつつある)
影響について論じなさい。(800字以内)
\end{quotation}

最近の技術に、
画像や動画の認識を行う
コンピュータビジョン技術として、
GoogleやAppleが提供する
Vision API
がある。

Googleは、主にクラウド側での画像認識機技術、
Appleは、端末側の技術として
提供している。

これは、
事前学習済の機械学習の利用により
実現可能となった
画像から物体や文字を取り出しモデリングする技術である。

スマートフォンのSoCの機械学習用プロセッサーやGPUの性能向上により、
高速かつ手軽に行えるようになってきた。

今後、LIDAR(レーザ画像検出と測距)と組み合わせることで、
精密に3Dモデリングができるようになると予測できる。

これは、自動運転に応用できるが、
身近なスマートフォンでも容易に、
3Dモデリングが可能となるため、
新たな応用が加わると予測できる。


\subsection{
脳コンピューター・インターフェイス(BCI)について
}

\begin{quotation}
マルチメデイアに関する最近の先端技術を1つ取りあげて解説し、
その技術的課題や応用を修士論文の課題として設定するとすれば、
どのようなテーマ、
研究方法が考えられるか述べなさい。
(800字以内)
\end{quotation}

脳コンピューター・インターフェイス(BCI)について

2020年8月29日、
イーロン・マスクは、
ニューラリンク社の
脳にチップを埋め込み、コンピュータと連動させる技術を発表した。
この技術は、以下となっている。

\begin{enumerate}
\item 1024個の極細の電極を、大脳新皮質(脳深部ではない)へ埋め込む
\item 脳波を記録する
\item 電極で脳を刺激することができる。
\item ワイアレスになっており、iPhoneで操作できる。
\item FDAへの申請が完了しており、2021年より人体へ適用できる。
\end{enumerate}

脳へ電極を埋め込み、
ロボットアームやコンピュータのカーソルを動かす試みは、
90年代後半からあった。
脳の挙動を観測するために、
生体への侵襲性がほとんどない脳波や、
脳の血流の画像採取(fMRI)が使われているが、
脳波の測定では、ノイズが多く、
採取できるデータの質に限界る等の問題があった。

この技術により、脳波のデータの質が改善することが期待できる。

また、最近の機械学習で使われるデータサイエンスの技術と組み合わせ、
採取した脳波データをと人間の動きを学習すれば、
有意な情報を取り出せるかもしれないが、
今のところ未開の領域である。

現状はあまり期待されている技術ではないという声もあるが、
2021年より人体へ適用可能となり、
採取できるデータも増えると思われるため、
大きな進展が起きるかもしれない。

修士論文の課題としては、BCIのデータを取得できるとして、
そのデータの分析を設定することができる。

\end{document}
