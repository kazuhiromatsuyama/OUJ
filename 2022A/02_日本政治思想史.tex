\documentclass{jsarticle}

\begin{document}

\title{本居宣長と平田篤胤の政治思想の共通点と相違点について}
\author{142-004814-1 松山和弘}
\maketitle

\section{本居宣長と平田篤胤の政治思想での位置付け}

% 全体の流れ
江戸時代の政治思想と宗教の出来事を順に並べていくと、
キリスト教の弾圧と寺檀制度の導入、
徳川家宣、徳川綱吉、新井白石による朱子学導入の試み、
陽明学や儒学者(伊藤仁斎,荻生徂徠)らによる朱子学批判、
本居宣長らによる国学の台頭(儒教の否定)、
平田篤胤らによる復古神道(国学の宗教化)、
水戸学(尊王攘夷)による天皇の祭祀化
という流れとなり、
儒教の導入と否定、キリスト教への対抗、神道の再構築の試みを見ることができる。

\section{本居宣長と平田篤胤の政治思想の共通点と相違点}

本居宣長は、
古事記や日本古典の研究を通して、
「天皇の地位が、アマテラスからの血統で保たれていること」が、
皇帝に対する革命を良しとする儒教思想よりも優れている主張した。
これにより、天皇の資質は不問であることとした。

平田篤胤は、
国学の宗教化を行い、
天皇も生前の行いにより、
キリスト教のように「死後、オオクニヌシの支配を受ける」とした。
これにより、天皇の資質を求めることとなった。

本居宣長と平田篤胤の共通点は、
儒教を否定する国学であることが共通である。

相違点は、宗教化(復古神道)と宗教化に伴う天皇の資質の扱いである。

\end{document}
