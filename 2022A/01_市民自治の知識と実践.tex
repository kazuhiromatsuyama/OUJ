

\documentclass{jsarticle}

\begin{document}

\title{市民自治の概念と市民自治を学ぶべき理由について}
\author{142-004814-1 松山和弘}
\maketitle

\section{市民自治の概念}

市民社会の概念を、概念史で分類すると、
古典的市民社会、
ブルジョア的市民社会、
結社的市民社会
がある。

アリストテレスは、
自らが所属するポリスの公的な職務(民会,陪審員,兵役)に積極的に参加することで、
はじめて市民になるとした。

日本国内の投票率の低さ(政治への関心の低さ)は、問題てあろう。

また、
兵役について思い当たるのが、
1793年 フランスの徴兵制導入により、基本的に傭兵を廃止したことである。
現代でも、志願制だと貧困層に兵役が偏る問題や、
軍事産業や傭兵企業の利害が戦争に影響する問題がある。
兵役の分担の公平性については考察する必要がある。
(古典的市民社会)

ヘーゲルは、市民社会を、
アダム・スミスの経済学を元に、
個人が私的利益のために、
市場を通して相互交流する経済社会とした。

日本は欧米と比較して、個人の金融資産の株式投資率が低い。
ここ30年間の国ごとの経済成長率、株価の格差もあり、
個人の市場参加率の低さと格差拡大が進行する状況となっている。
こりについても考察する必要がある。
(ブルジョア的市民社会)

1980年代東欧の民主主義革命
(ポーランドの独立自主管理労働組合「連隊」)
のような、(非民主的な)国家と対抗的な政治性をもつ市民社会
(結社的市民社会)

2022年2月24日より、ロシアによるウクライナ侵攻が行われている。
また、プーチンによる核兵器の使用の可能性がある状況となっている。
これまで、西側の保守派には、
プーチンのような独裁者を容認するような立場があったが、
現在では否定されている。
もし、ロシアがウクライナに勝つと、それが世界の秩序となり、
自由と民主主義は損なわれる恐れがあるため、
支援が必要となった。



\section{市民自治を学ぶべき理由}

合成の誤謬

\end{document}

