

\documentclass{jsarticle}

\begin{document}

\title{市民自治の概念と市民自治を学ぶべき理由について}
\author{142-004814-1 松山和弘}
\maketitle

\section{市民自治の概念}

市民社会の概念を、概念史で分類すると、
古典的市民社会、
ブルジョア的市民社会、
結社的市民社会
がある。

アリストテレスは、
自らが所属するポリスの公的な職務(民会,陪審員,兵役)に積極的に参加することで、
はじめて市民になるとした(古典的市民社会)。
現在の問題としては、日本国内の投票率の低さ(政治への関心の低さ)がある。
また、兵役については、志願制だと貧困層に兵役が偏る問題や、
軍事産業や傭兵企業の利害が戦争に影響する問題があり、
兵役の負担が公平ではないことが問題となっている。

ヘーゲルは、市民社会を、
アダム・スミスの経済学を元に、
個人が私的利益のために、
市場を通して相互交流する経済社会とした
(ブルジョア的市民社会)。
日本は欧米と比較して、個人の金融資産の株式投資率が低い。
結果的にここ30年間は、の国ごとの経済成長率が影響するが、
欧米と日本の経済的格差が拡大している。
個人の市場参加率の低さが、
格差拡大の要因の一つと考えられるが、国内ではあまり問題とはなってないようだ。

1980年代東欧の民主主義革命
(ポーランドの独立自主管理労働組合「連隊」)
のような、(非民主的な)国家と対抗的な政治性をもつ市民社会がある。
(結社的市民社会)

%2022年2月24日より、ロシアによるウクライナ侵攻が行われている。
%また、プーチンによる核兵器の使用の可能性がある状況となっている。
%これまで、西側の保守派には、
%プーチンのような独裁者を容認するような立場があったが、
%現在では否定されている。
%もし、ロシアがウクライナに勝つと、それが世界の秩序となり、
%自由と民主主義は損なわれる恐れがあるため、
%支援が必要となった。


\section{市民自治を学ぶべき理由}

市民自治を学ぶべき理由に、
「合成の誤謬」等により、
「意図せざる結果」
となることを意識するべきであるからである。

例えば、些細な失言を問題にして政治家を失脚させてしまった後、
筋の悪い政治家を選んでしまうケースは多くある。

筋の悪い政治家とは、
極右やカルト宗教とを動員して政治的圧力をかけたり、
問題のある金融政策を行なったり、
公文書や社会統計を改竄したり、
不正会計を行う
者のことである。


\end{document}

