
\documentclass{jsarticle}

\begin{document}

\title{現代の政治理論の営み}
\author{141-821706-0 松山和弘}
\maketitle

2022年2月24日に、ロシアによるウクライナ侵攻がはじまった。
近隣諸国に500万人を超える難民が流れ込み、
西欧、北欧の安全保障の政治戦略が大きく変わり、ロシアが勝利することを阻止する方向となっている。
そして、ロシアからの天然ガスや石油の供給の停止によるエネルギー危機、
世界的な穀物の需給関係の崩による食糧危機を恐れる状況となっている。
また、プーチンによる明示的な核使用の威嚇があり、人類存亡の危機的な状況となっている。

政治理論には、
現実の政治を記述・分析する、経験的政治理論と、
政治のあるべき姿を描き、その妥当性を検証する、
規範的政治理論があるが、
人類存亡の危機により、経験的に政治理論うことができなくなるかもしれない
状況となっている。
以上により、政治理論を考る上で、ウクライナ侵攻は除外できないと考える。

さて、政治理論の営みをロールズを範にみていくことにする。
ロールズは、「民主主義」の立場で、「社会契約論」を元に、
「功利主義」を批判して「正義論(1971年)」を展開する。
基本的な社会・経済・政治構造の上に「正義の原理」があるとした。
当時、公民権運動、ベトナム戦争などのアメリカの危機が問題となっていた。
後期ロールズは、「資本主義経済」は正義ではないとし、
「事後的な再分配による福祉資本主義」を否定した。
当時は、グローバル経済の拡大やネオリベラリズムへの偏りが影響していると考えられる。
ここには、米国の問題から規範的な政治理論を構築しようとする営みを見ることができる。

ロールズの、現代的な役割をどう考えればよいだろうか
「事後的な再分配による福祉資本主義」を正義ではないとして、
排除しても良いのか?
という問題がある。

ロールズの「労働者管理型企業」について考えてみたい。
米国企業では、企業年金制度として、401kの加入が増加した。
これらの多くは株式に投資しており、
結果的に労働者の資本蓄積が増大することとなった。
また、年金基金などの企業へのガバナンスの適正化の要求が強くなって
おり、発言力が強くなっている。

\end{document}



