
\documentclass{jsarticle}

\begin{document}

\title{4世紀から12世紀ヨーロッパの聖俗関係について}
\author{141-821706-0 松山和弘}
\maketitle

800年12月25日に、
カール大帝は教皇レオ3世からローマ皇帝の冠を受けた。
皇帝の戴冠を教皇が行うことで、
型式上、皇帝(世俗権力)は教会への従属することとなった。
また、これが東方教会とは異なる西方教会の独自の方向性となった。
その後、ミサの典礼文の変更や、ローマ司教首位の扱いなど西方教会
と東方教会の差異が問題となり、
1054年に東西教会の分裂となった。

聖職者が権力を持つようになったが、
11世紀頃までに腐敗が進みシモニア(聖職売買)や、
聖職者の妻帯が行われるようになった。

これに対し、まず世俗権力のハインリヒ3世による
シモニア改革が行われ、シモニアに関わった教皇3名が退位させられた。

世俗の権力者が教皇の人事をおこなったことに対する批判が、
グレゴリウス7世らによる、
教皇改革 (教皇の選挙制度改革、教皇庁への権力集中、叙任権闘争) 
へとつながっていき、今日まで続く教会法の整備や
教皇の選挙制度が確立する。
教皇の選挙制度(コンクラーベ)は、
世俗の権力からの介入を阻止する仕組みとして興味深い。

\end{document}



